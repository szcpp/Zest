\documentclass[11pt,a4paper]{article} 

%lokalizacja
\usepackage{polski}
\usepackage[utf8]{inputenc}
\usepackage[OT4]{fontenc}

%dodatkowe pakiety
\usepackage{indentfirst}
\usepackage{enumerate} %listowanie
\usepackage[dvips]{graphics} %grafiki
\usepackage{graphicx}
\usepackage{amsmath} %pakiet potrzebny do \eqref
\usepackage{tabularx} %tabele

%informację o dokumencie
\author{Anna Zaborowska, Mateusz Gałażyn}
\title{Projekt ZEST}

%marginesy
%\addtolength{\textwidth}{4cm}
%\addtolength{\hoffset}{-2cm}
%\addtolength{\textheight}{4cm}
%\addtolength{\voffset}{-2cm}

\begin{document}
\maketitle

\begin{center}
\Large {\textsl{Dokumentacja końcowa}}
\end{center}

\section{Krótki opis projektu}

Projekt Zest jest komunikatorem służącym do bezpiecznej wymiany informacji pomiędzy użytkownikami. Komnunikacja odbywa się w trybie peer to peer i jest szyfrowana za pomocą szyfru asymetrycznego GPG (z kluczem publicznym).

\section{Zastosowane zaawansowane techniki C++}

\begin{itemize}
\item[-] kontenery STL (std::vector);
\item[-] wzorzec projektowy: Singleton (klasa Interface);
\item[-] wzorzec projektowy: Obserwator (klasa Mesasge,  P2PConnection);
\item[-] wątki (Contact);
\item[-] wyjątki (P2PConnection::xConnectionFailure);
\item[-] biblioteka ncurses (w tym obsługa klawiatury);
\end{itemize}

\section{Podział na jednostki kompilacyjne, opis}
\subsection{Observer}
Implementacja wzorca projektowego Obserwator.

\subsection{Observable}
Implementacja wzorca projektowego Obserwator.

\subsection{P2PConnection}
Reprezentuje pojedyncze połączenie użytkownika, posiada metody pozwalające na odbieranie i wysyłanie wiadomości.

\subsection{P2PServer}
Odpowiedzialny za otrzymywanie nowych połączeń.

\subsection{Message}
Reprezentuje wiadomość wymienianą pomiędzy użytkownikami, zawiera datę oraz adres IP nadawcy.

\subsection{Contact}
Reprezentuje pojedynczego klienta, nasłuchuje nowe wiadomości, zmienia status użytkownika lub drukuje na ekran wiadomość jednocześnie dodając ją do wektora wiadomości.

\subsection{Interface}
Klasa spinająca wszystkie elementy interfejsu a także przechowywująca wektor listy kontaktów, zawiaduje otwartymi połączeniami, decyduje gdzie umieścić dostarczone wiadomości, reaguje na zmianę rozmiaru terminala.

\subsection{ContactList}
Panel wyświetlający listę kontaktów.

\subsection{ChatWindow}
Panel wyświetlający treść rozmowy z użytkownikiem. Odświeżany po wysłaniu wiadomości bądź jej otrzymaniu. 

\subsection{InputField}
Pole tekstowe do wpisywania treści wiadomości.

\subsection{InterfaceIndicator}
Wyświetla otwarte okna rozmów, pogrubione zostaje aktywne okno.

\subsection{MessagesList}
Zawiera wektor wiadomości.


\end{document}