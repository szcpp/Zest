\documentclass[11pt,a4paper]{article} 

%lokalizacja
\usepackage{polski}
\usepackage[utf8]{inputenc}
\usepackage[OT4]{fontenc}

%dodatkowe pakiety
\usepackage{indentfirst}
\usepackage{enumerate} %listowanie
\usepackage[dvips]{graphics} %grafiki
\usepackage{graphicx}
\usepackage{amsmath} %pakiet potrzebny do \eqref
\usepackage{tabularx} %tabele

%informację o dokumencie
\author{Anna Zaborowska, Mateusz Gałażyn}
\title{Projekt ZEST}

%marginesy
%\addtolength{\textwidth}{4cm}
%\addtolength{\hoffset}{-2cm}
%\addtolength{\textheight}{4cm}
%\addtolength{\voffset}{-2cm}

\begin{document}
\maketitle

\begin{center}
\Large {\textsl{Podręcznik użytkownika}}
\end{center}

\section{Krótki opis projektu}

Projekt Zest jest komunikatorem służącym do bezpiecznej wymiany informacji pomiędzy użytkownikami. Komnunikacja odbywa się w trybie peer to peer i jest szyfrowana za pomocą szyfru asymetrycznego GPG (z kluczem publicznym).

\section{Uruchamianie programu}

Do działania programu niezbędna jest biblioteka ncurses.h, a ponadto dwie dodatkowe z nią skojarzone: panel.h oraz menu.h. Program jest kompilowany za pomocą komendy ,,make'', po czym może zostać uruchomiony poleceniem ,,./zest''.

\section{Obsługa programu}

Działanie komunikatora opiera aię na stosowaniu prostych skrótów klawiszowych:
\begin{itemize}
\item[ ENTER ] wysyła wiadomość wpisaną wcześniej w polu wiadomości;
\item[ KEY\_LEFT ] nawigacja po liście kontaktów; 
\item[ KEY\_RIGHT ] nawigacja po liście kontaktów; 
\item[ KEY\_UP ] przesuwanie okna wiadomości (dla liczby wiadomości przekraczającej pojemność okna);
\item[ KEY\_DOWN ] przesuwanie okna wiadomości (dla liczby wiadomości przekraczającej pojemność okna);
\item[ + ] otwarcie nowego okna rozmowy dla aktualnie podświetlonego kontaktu;
\item[ TAB ] przechodzenie do innego okna rozmowy;
\item[ ESC ] wyjście z programu;
\end{itemize}

\end{document}