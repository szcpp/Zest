\documentclass[11pt,a4paper]{article} 

%lokalizacja
\usepackage{polski}
\usepackage[utf8]{inputenc}
\usepackage[OT4]{fontenc}

%dodatkowe pakiety
\usepackage{indentfirst}
\usepackage{enumerate} %listowanie
\usepackage[dvips]{graphics} %grafiki
\usepackage{graphicx}
\usepackage{amsmath} %pakiet potrzebny do \eqref
\usepackage{tabularx} %tabele

%informację o dokumencie
\author{Anna Zaborowska, Mateusz Gałażyn}
\title{Projekt ZEST}

%marginesy
%\addtolength{\textwidth}{4cm}
%\addtolength{\hoffset}{-2cm}
%\addtolength{\textheight}{4cm}
%\addtolength{\voffset}{-2cm}

\begin{document}
\maketitle

\section{Krótki opis projektu}

Projekt zest będzie komunikatorem służącym do bezpiecznej wymiany informacji pomiędzy użytkownikami. Komnunikacja będzie odbywała się w trybie peer to peer i będzie szyfrowana za pomocą szyfru asymetrycznego GPG (z kluczem publicznym).

\section{Zarys stosowanych technik}

\begin{itemize}
\item[-] kontenery STL (vector - lista kontaktów);
\item[-] algorytmy STL (sortowanie listy kontaktów, statystyka listy kontaktów;
\item[-] wzorzec projektowy: Singleton (interfejs);
\item[-] wzorzec projektowy: Obserwator (interakcja pomiędzy klasami);
\item[-] wątki;
\item[-] wyjątki;
\item[-] biblioteka ncurses (w tym obsługa klawiatury);
\end{itemize}

\section{Podział na jednostki kompilacyjne, opis}
\subsection{Observer}
Implementacja wzorca projektowego Obserwator.

\subsection{Observable}
Implementacja wzorca projektowego Obserwator.

\subsection{P2PConnection}
Reprezentuje pojedyncze połączenie użytkownika, posiada metody pozwalające na odbieranie i wysyłanie wiadomości.

\subsection{P2PServer}
Odpowiedzialny za otrzymywanie nowych połączeń.

\subsection{Message}
Reprezentuje wiadomość wymienianą pomiędzy użytkownikami, jednocześnie informuje o statusie (dostępności) użytkownika.

\subsection{Contact}
Reprezentuje pojedynczego klienta, umożliwia uaktualnienie jego statusu.

\subsection{Interface}
Klasa spinająca wszystkie elementy interfejsu.

\subsection{ContactList}
Panel wyświetlający listę kontaktów. Po uruchomieniu programu odpytuje wszystkie kontakty, potem odświeża statusy przy ich zmianie. 

\subsection{ChatWindow}
Panel wyświetlający treść rozmowy z użytkownikiem. Odświeżany po wysłaniu wiadomości bądź jej otrzymaniu. 

\subsection{InputField}
Pole tekstowe do wpisywania treści wiadomości.

\end{document}